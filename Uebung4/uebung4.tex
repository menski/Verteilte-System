\documentclass[german,12pt,a4paper]{article}
\usepackage{fullpage}
\usepackage[ngerman]{babel}
\usepackage[utf8]{inputenc}
\usepackage{listings}
\usepackage{verbatim}
\usepackage{enumerate}
\usepackage{graphicx}
\usepackage{float}
\usepackage{wrapfig}
\usepackage{color}
\usepackage[usenames,dvipsnames]{xcolor}
\usepackage[font=small,format=plain,labelfont=bf,up,textfont=it,up]{caption}
\usepackage{subfig}
\usepackage[colorlinks=false, pdfborder={1 0 0}]{hyperref}
\usepackage{qtree}


\pagestyle{plain}
\pagenumbering{arabic}
\frenchspacing

\newcommand{\comments}[1]{}
\renewcommand{\baselinestretch}{1.55}

%Redefine the first level
\renewcommand{\theenumi}{\textbf{\alph{enumi})}}
\renewcommand{\labelenumi}{\theenumi}

\begin{document}

\title{\textbf{Verteilte Systeme SS2012 -- Übung 4}}
\author{Sebastian Menski (734272), Martin Ohmann (734801) \\ \texttt{\{menski,ohmann\}@uni-potsdam.de}}
\date{\today}

\maketitle

\section*{Aufgabe 4.1}

\begin{enumerate}

	\item
		\begin{tabular}{ccc}
			Router & Hops & Gateway \\\hline
			B & 1 & B \\
			C & 2 & F \\
			D & 2 & B \\
			E & 1 & E \\
			F & 1 & F \\
			G & 2 & E \\
			H & 2 & F \\
			I & 3 & B (oder F)\\
		\end{tabular}
	
	\item Es treten 16 Hops auf. Zuerst sendet A $\rightarrow$ (B, F, E); dann B $\rightarrow$ (D, F), F $\rightarrow$ (B, C, H), E
		$\rightarrow$ (F, H); dann D $\rightarrow$ (C, I), C $\rightarrow$ D, H $\rightarrow$ (G, I), G
		$\rightarrow$ H.\\Daraus ergibt sich folgender Baum:
		\Tree [.A [.B [.D I ] ] [.F C H ] [.E G ] ]

	
	\item Text

\end{enumerate}

\section*{Aufgabe 4.2}

Text

\section*{Aufgabe 4.3}

\begin{enumerate}

	\item Der Nutzer kann durch NFS auf nicht lokale Dateien zugreifen, diese werden jedoch direkt in
		das lokale Filesystem des Users eingebunden. Somit wird eine Ortstransparenz erreicht.
	
	\item Der Administrator muss die Entfernten Filesystems und ihre lokalen Mount-Punkte dem System
		bekannt machen (meißt in die \texttt{/etc/fstab} eintragen).
	
	\item NFSv3 war ein zustandsloser Dienst, d.h. der Client musste bei jedem Datenaustausch alle
		benötigten Parameter mitsenden. NFSv4 hingegegen ist ein zustandsbehafteter Dienst, so mit
		existiert für jeden Client ein Zustand auf dem Server, wodurch nun auch \texttt{open} und
		\texttt{close} Befehle verfügbar sind.
	
	\item 5? (Für jeden Teil des Pfads einen?)
	
	\item Der Server hält den State des Clients für die Lease-Time. Wenn der Client nicht innerhalb
		der Lease-Time diese erneuert, wird der State des Clients vom Server verworfen.

\end{enumerate}

\end{document}

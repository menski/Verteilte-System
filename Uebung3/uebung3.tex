\documentclass[german,12pt,a4paper]{article}
\usepackage{fullpage}
\usepackage[ngerman]{babel}
\usepackage[utf8]{inputenc}
\usepackage{listings}
\usepackage{verbatim}
\usepackage{enumerate}
\usepackage{graphicx}
\usepackage{float}
\usepackage{wrapfig}
\usepackage{color}
\usepackage[usenames,dvipsnames]{xcolor}
\usepackage[font=small,format=plain,labelfont=bf,up,textfont=it,up]{caption}
\usepackage{subfig}
\usepackage[colorlinks=false, pdfborder={1 0 0}]{hyperref}


\pagestyle{plain}
\pagenumbering{arabic}
\frenchspacing

\newcommand{\comments}[1]{}
\renewcommand{\baselinestretch}{1.55}

%Redefine the first level
\renewcommand{\theenumi}{\textbf{\alph{enumi})}}
\renewcommand{\labelenumi}{\theenumi}

\begin{document}

\title{\textbf{Verteilte Systeme SS2012 -- Übung 3}}
\author{Sebastian Menski (734272), Martin Ohmann (734801) \\ \texttt{\{menski,ohmann\}@uni-potsdam.de}}
\date{\today}

\maketitle

\section*{Aufgabe 3.1}

\begin{enumerate}

	\item Der UDP-Server verwendet vier Threads, wovon die UDPListenerRunnables für Unicast, Multicast und Bradcast jeweils einen nutzen.
	Der vierte Thread ist der UDPServer-Thread selbst. Zur Lastermittlung wurde \texttt{htop} benutzt. Auffällig ist, dass die Lastverteilung hierbei sehr unausgeglichen ist: der UDPServer-Thread nutzt 99-100\% 
	der CPU, die Threads der UDPListenerRunnables zeigen beim Senden einer Nachricht vom Client an den Server keine Veränderung der Lastverteilung.
	
	\item Die hohe CPU-Auslastung durch den UDPServer- Thread wird durch eine leere \texttt{while(true)}-Schleife in der \texttt{run}-Methode des UDPServers verursacht. 
	Dies lässt sich durch ein \texttt{Thread.sleep()} im Schleifenkörper beheben. Nach dem Bugfix lässt sich erkennen, dass beim Senden einer Nachricht vom Client an den Server die Auslastung des UDPServer-Threads bei 
	0\% liegt, die Auslastung des jeweiligen UDPListenerRunnable-Threads jedoch kurzzeitig auf etwa 1\% steigt.

\end{enumerate}

\section*{Aufgabe 3.2}

\begin{enumerate}

	\item Broadcast und Multicast haben die gemeinsame Eigenschaft, Packete an mehrere Empfänger gleichzeitig zu versenden. 
	Der Unterschied zwischen Multicast und Broadcast besteht darin, dass die mittles Broadcast 
	gesendeten Packete vom jeden Teilnehmer des Netzwerks empfangen werden können, wohingegen beim 
	Multicast eine vorherige Anmeldung beim Sender der Packete erforderlich ist.
	 
	\item In IPv4 wird in zwei verschiedene Broadcast-Formen unterschieden:
		\begin{itemize}
			\item \textbf{Limited Braodcast}: Hier wird als Ziel die IP-Adresse \texttt{255.255.255.255} angegeben. Dieses befindet sich 
			immer im eigenen Netz und wird direkt als Ethernet-Broadcast umgesetzt. Es findet keine Weiterleitung durch einen Router 
			statt.
			\item \textbf{Directed Broadcast}: Beim \textit{Directed Broadcast} werden Packete an Teilnehmer eines bestimmten Netzes versendet.
			Für einen \textit{Directed Broadcast} in das Netz \texttt{192.168.0.0} mit der Netzmaske \texttt{255.255.255.0} lautet die Zieladresse 
			\texttt{(192.168.0.0/24):192.168.0.255}. \textit{Directed Broadcasts} werden von einem Router weitergeleitet, falls Quell- und 
			Zielnetz unterschiedlich sind. Sind die Netze identisch, so entspricht ein \textit{Directed Broadcast} dem \textit{Limited Broadcast}.
		\end{itemize}
		
		
		Für Multicast in IPv4 ist der Adressbereich von \texttt{224.0.0.0} bis \texttt{239.255.255.255} reserviert. Bei einem 
		Mulicast wird die Multicast-Adresse auf eine Pseudo-MAC-Adresse abgebildet, um der Netzwerkkarte die Filterung nach relevanten 
		Packeten zu ermöglichen. Dazu werden die untersten 23 Bit der IP-Adresse in die Mac-Adresse \texttt{01-00-5e-00-00-00} eingesetzt, 
		wodurch sich MAC-Adressen im Bereich \texttt{01-00-5e-00-00-00} bis \texttt{01-00-5e-7f-ff-ff} ergeben können.
		

	\item 
		\begin{itemize}
			\item \textbf{Audio-/Videotelefonat}: Hier wird Multicast genutzt. Dies ist sinnvoll, da beim Multicast die Packete nur jeweils 
			einmal versendet werden, was den verursachten Traffic minimiert.
			\item \textbf{Chat}: 
			\item \textbf{sonstige Nachrichten}:
		\end{itemize}
	
\end{enumerate}

\section*{Aufgabe 3.3}

\begin{enumerate}

	\item Der Begriff \textit{Remote Procedure Call} bezeichnet eine Technik, welche zur Interprozesskommunikation genutzt wird. 
	Dabei wird der Aufruf von Funktionen in anderen Adressräumen, wie etwa entfernten Rechnern, ermöglicht. In der Regel wird die 
	aufgerufene Funktion auf einen anderen Rechner als dem des aufrufenden Programms ausgeführt.
	 
	\item blah

	\item blah
	
\end{enumerate}

\section*{Aufgabe 3.5}

\begin{enumerate}

	\item Der Client kontaktiert einen Rechner, welcher zwar im Netzwerk erreichbar ist, jedoch den RMI-Service nicht gestartet hat.
	Der Client wirft eine ConnectionException, da die Verbindung vom Server abgewiesen wird.
	 
	\item Der Client versucht sich mit einer nicht erreichbaren/nicht vergebenen Netzwerkadresse zu verbinden. 
	Der Client wirft einen UnknownHostException, da er den Hostnamen nicht auflösen kann.

	\item Der Server wird dahingehend manipuliert, sodass er in einer Endlosschleife hängt. Der Client sendet nun eine Anfrage an 
	den Server, auf welche dieser nicht antwortet. Dier Client wartet in diesem Fall unendlich lang auf die Antwort des Servers, da standardmäßig 
	kein Timeout gesetzt ist. Dieses kann man jedoch über das tcp-responseTimeout selbst festlegen, um einem Hängen des Clients durch einen 
	nicht reagierenden Server vorzubeugen.
	
\end{enumerate}

\end{document}

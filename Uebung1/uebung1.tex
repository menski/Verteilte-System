\documentclass[german,12pt,a4paper]{article}
\usepackage{fullpage}
\usepackage[ngerman]{babel}
\usepackage[utf8]{inputenc}
\usepackage{listings}
\usepackage{verbatim}
\usepackage{enumerate}
\usepackage{graphicx}
\usepackage{float}
\usepackage{wrapfig}
\usepackage{color}
\usepackage[usenames,dvipsnames]{xcolor}
\usepackage[font=small,format=plain,labelfont=bf,up,textfont=it,up]{caption}
\usepackage{subfig}

\pagestyle{plain}
\pagenumbering{arabic}
\frenchspacing

\newcommand{\comments}[1]{}
\renewcommand{\baselinestretch}{1.55}

%Redefine the first level
\renewcommand{\theenumi}{\textbf{\alph{enumi})}}
\renewcommand{\labelenumi}{\theenumi}

\begin{document}

\title{\textbf{Verteilte Systeme SS2012 -- Übung 1}}
\author{Sebastian Menski (734272), Martin Ohmann (734801) \\ \texttt{\{menski,ohmann\}@uni-potsdam.de}}
\date{\today}

\maketitle

\section*{Aufgabe 1.1}

\begin{description}
	\item[Primäre Transparenzeigenschaften] Text
	\item[Sekundäre Transparenzeigenschaften] Text
\end{description}


\section*{Aufgabe 1.2}

\begin{enumerate}
	\item 
		\begin{description}
			\item[\texttt{scp}] ist ein Tool zur verschlüsselten Datenübertragung 
				zwischen zwei Computern innerhalb eines Rechnernetzes.
			\item[\texttt{rdist}] ist dazu gedacht, identische Kopien von Dateien 
				auf mehreren Hosts zu verwalten. Dateiattribute wie Besitzer, Gruppen, 
				Dateirechte und Änderungszeit werden dabei wenn möglich beibehalten.
			\item[\texttt{rsync}] ist ein Programm zur Synchronisation von Daten, 
				welches meist über ein Netzwerk genutzt wird. \texttt{rsync} arbeitet
				unidirektional, das heißt, dass die Daten vom 
				Quellverzeichnis hin zum Zielverzeichnis synchronisiert werden. Wurden 
				Dateien im Zielverzeichnis geändert, so gehen diese Änderungen verloren.
				Desweiteren besitzt \texttt{rsync} die Fähigkeit Teile von Dateien 
				zu kopieren. Bei Änderungen an einer Datei im Quellverzeichnis werden 
				nur die veränderten Teile dieser übertragen, was eine erhebliche Einsparung von 
				Traffic zur Folge haben kann.
		\end{description}
	\item Zu Abgleich von Dateien zwischen einem mobilen Client und einem Fileserver 
		eignet sich \texttt{rdist} am besten, da hier die Daten, anders als bei \texttt{scp},
		und \texttt{rsync}, in beide Richtungen synchronisiert werden.
\end{enumerate}

\end{document}

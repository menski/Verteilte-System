\documentclass[german,12pt,a4paper]{article}
\usepackage{fullpage}
\usepackage[ngerman]{babel}
\usepackage[utf8]{inputenc}
\usepackage{listings}
\usepackage{verbatim}
\usepackage{enumerate}
\usepackage{graphicx}
\usepackage{float}
\usepackage{wrapfig}
\usepackage{color}
\usepackage[usenames,dvipsnames]{xcolor}
\usepackage[font=small,format=plain,labelfont=bf,up,textfont=it,up]{caption}
\usepackage{subfig}

\pagestyle{plain}
\pagenumbering{arabic}
\frenchspacing

\newcommand{\comments}[1]{}
\renewcommand{\baselinestretch}{1.55}

%Redefine the first level
\renewcommand{\theenumi}{\textbf{\alph{enumi})}}
\renewcommand{\labelenumi}{\theenumi}

\begin{document}

\title{\textbf{Verteilte Systeme SS2012 -- Übung 1}}
\author{Sebastian Menski (734272), Martin Ohmann (734801) \\ \texttt{\{menski,ohmann\}@uni-potsdam.de}}
\date{\today}

\maketitle

\section*{Aufgabe 1.1}

\begin{itemize}
	\item \textbf{Primäre Transparenzeigenschaften} sollen die Verteiltheit des Systems gegenüber dem Benutzer
	verdecken. Dabei handelt es sich um folgende Transparenzen:
	\begin{itemize}
		\item \textbf{Ortstransparenz} steht dafür, dass der Nutzer nicht den genauen Ort der Ressource kennen
		muss, das bedeutet:
		\begin{itemize}
			\item \textbf{Zugriffstransparenz}: Der Zugriff auf einen lokalen oder entfernten Webserver erfolgt
			immer mit den selben Programmen (z.B. Webbrowser, ssh, telnet).
			\item \textbf{Namenstransparenz}: Ist lokal eine Replikation eines entfernten Webservers eingerichtet,
			so kann auf alle Ressourcen über denselben Pfad, abgesehen vom Hostname, zugegriffen werden. Dabei
			ist es für den Nutzer nicht von Bedeutung wo diese Ressourcen genau auf dem Dateisystem liegen.
		\end{itemize}
		\item \textbf{Skalierungstransparenz} bedeutet, dass der Nutzer eine Erweiterung des Systems nicht
		bemerkt oder davon beeinträchtigt wird. Das heißt für einen Webserver, dass dieser trotz
		Erweiterung des Systems weiterhin alle Dienste anbietet und nach der Erweiterung im besten Fall
		leistungsfähiger ist.
	\end{itemize}
	\item \textbf{Sekundäre Transparenzeigenschaften} sollen die Verteiltheit des Systems zur Steigerung der
	Performance und der Fehlertoleranz ausnutzen. Dabei handelt es sich um folgende Transparenzen:
	\begin{itemize}
		\item \textbf{Replikationstransparenz} soll die Verfügbarkeit des Systems durch die Replikation
		der Ressourcen erhöhen. So merkt ein Benutzer eines Webservers nicht, ob er mit einer einzelnen
		Machine kommuniziert oder mit einem ganzes Cluster von Webservern.
		\item \textbf{Migrationstransparenz} soll die Verlagerung von Prozessen oder das Verschieben von
		Ressourcen vor dem Nutzer verbergen. So sollte ein Nutzer es nicht merken, falls auf eine andere
		Webserver-Software gewechselt wird oder die Ressourcen auf dem Webserver lokal verschoben
		werden.
		\item \textbf{Fehlertransparenz} soll verhindern, dass Fehler für Nutzer sichtbar sind. So sollte
		z.B. ein Webserver-Cluster den Ausfall eines Servers selbst auffangen, ohne dass der Nutzer dies
		bemerkt.
		\item \textbf{Nebenläugkeitstransparenz} soll den parallelen Zugriff auf gemeinsame Ressourcen
		synchronisieren. Nutzer sollen nicht bemerken, wenn anderen Nutzer ebenfalls dieselbe
		Ressource des Webservers anfordern. Bei verändernden Aktivitäten auf dem Webserver, muss dieser
		dafür sorgen, dass es zu keinen Konflikten kommt. 
	\end{itemize}
\end{itemize}


\section*{Aufgabe 1.2}

\begin{enumerate}
	\item 
		\begin{description}
			\item[\texttt{scp}] ist ein Tool zur verschlüsselten Datenübertragung 
				zwischen zwei Computern innerhalb eines Rechnernetzes.
			\item[\texttt{rdist}] ist dazu gedacht, identische Kopien von Dateien 
				auf mehreren Hosts zu verwalten. Dateiattribute wie Besitzer, Gruppen, 
				Dateirechte und Änderungszeit werden dabei wenn möglich beibehalten.
			\item[\texttt{rsync}] ist ein Programm zur Synchronisation von Daten, 
				welches meist über ein Netzwerk genutzt wird. \texttt{rsync} arbeitet
				unidirektional, das heißt, dass die Daten vom 
				Quellverzeichnis hin zum Zielverzeichnis synchronisiert werden. Wurden 
				Dateien im Zielverzeichnis geändert, so gehen diese Änderungen verloren.
				Desweiteren besitzt \texttt{rsync} die Fähigkeit Teile von Dateien 
				zu kopieren. Bei Änderungen an einer Datei im Quellverzeichnis werden 
				nur die veränderten Teile dieser übertragen, was eine erhebliche Einsparung von 
				Traffic zur Folge haben kann.
		\end{description}
	\item Zum Abgleich von Dateien zwischen einem mobilen Client und einem Fileserver 
		eignet sich \texttt{rdist} am besten, da hier die Daten, anders als bei \texttt{scp},
		und \texttt{rsync}, in beide Richtungen synchronisiert werden.
\end{enumerate}

\end{document}

\documentclass[german,12pt,a4paper]{article}
\usepackage{fullpage}
\usepackage[ngerman]{babel}
\usepackage[utf8]{inputenc}
\usepackage{listings}
\usepackage{verbatim}
\usepackage{enumerate}
\usepackage{graphicx}
\usepackage{float}
\usepackage{wrapfig}
\usepackage{color}
\usepackage[usenames,dvipsnames]{xcolor}
\usepackage[font=small,format=plain,labelfont=bf,up,textfont=it,up]{caption}
\usepackage{subfig}
\usepackage[colorlinks=false, pdfborder={1 0 0}]{hyperref}


\pagestyle{plain}
\pagenumbering{arabic}
\frenchspacing

\newcommand{\comments}[1]{}
\renewcommand{\baselinestretch}{1.55}

%Redefine the first level
\renewcommand{\theenumi}{\textbf{\alph{enumi})}}
\renewcommand{\labelenumi}{\theenumi}

\begin{document}

\title{\textbf{Verteilte Systeme SS2012 -- Übung 2}}
\author{Sebastian Menski (734272), Martin Ohmann (734801) \\ \texttt{\{menski,ohmann\}@uni-potsdam.de}}
\date{\today}

\maketitle

\section*{Aufgabe 2.1}

\begin{enumerate}

	\item Das Interface eth0 ist direkt an das private Netzwerk \texttt{10.3.5.0/24} angeschlossen und
	besitzt eine IP Adresse in diesem Netzwerk. Daher soll alle Paket, welche ebenfalls an eine
	Adresse in diesem Netzwerk gesendet werden über das Interface eth0 verarbeitet werden. Der 2.
	Eintrag, beginnend mit \texttt{0.0.0.0} steht für die \textit{default} Route. Das bedeutet, dass
	alle Paket, welche nicht an eine Adresse in dem durch die 1. Regel abgedeckten Netzwerk gesendet
	werden, an den Standard Gateway mit der Adresse \texttt{10.3.5.254} gesendet werden, welcher
	ebenfalls über das Interface eth0 erreichbar ist.

	\item Beide Programme nutzen ICMP Nachrichten. Ping sendet ein \textit{Echo Request}-Paket und
	erhält entsprechende Antwort ob ein Rechner verfügbar ist. Traceroute ermittelt alle Knoten auf
	der Route zum Ziel. Dazu werden ebenfalls mehrere ICMP-Paket gesendet, jedoch wird die Funktion
	der TTL genutzt. Das heißt, das erste gesendete Paket enthält eine TTL von 1, weshalb es schon vom
	ersten Knoten auf der Route verworfen wird und dieser eine Fehlermeldung mit seiner Adresse
	zurückschickt. Dies wird mit steigender TTL wiederholt bis das Ziel erreicht wurde. Somit können
	alle Knoten auf der Route erkannt werden.

	\item Mit dem Programm \texttt{route} können die IP-Routing-Tabellen des Kernels verändert werden,
	somit der komplette Netzwerk-Traffics des Rechners umgeleitet werden, was nicht jedem Nutzer
	möglich sein sollte. Mit dem Programm \texttt{ifconfig} kann zum Konfigurieren der
	Netzwerkschnittstellen genutzt werden, welches ebenfalls eine Manipulation oder Störung des
	gesamten Netzwerk-Traffics ermöglicht, was ebenfalls nicht jedem Nutzer möglich sein sollte.

\end{enumerate}


\section*{Aufgabe 2.2}
\begin{table}[h]
\begin{tabular}{|c|c|r|r|r|}\hline
Netzklasse & Adressbereich               & Hostlänge & Netze      & Hosts pro Netz \\\hline
Klasse A   &   0.0.0.0 – 127.255.255.255 & 24 Bit    &	      128	& 16.777.214     \\
Klasse B   & 128.0.0.0 – 191.255.255.255 & 16 Bit    &	   16.384 &	    65.534     \\
Klasse C   & 192.0.0.0 – 223.255.255.255 & 	8 Bit    &	2.097.152	&        254     \\\hline
\end{tabular}
\caption{Quelle: \url{http://de.wikipedia.org/wiki/Netzklasse}}
\end{table}


\section*{Aufgabe 2.3}

\section*{Aufgabe 2.4}

\end{document}

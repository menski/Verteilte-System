\documentclass[german,12pt,a4paper]{article}
\usepackage{fullpage}
\usepackage[ngerman]{babel}
\usepackage[utf8]{inputenc}
\usepackage{listings}
\usepackage{verbatim}
\usepackage{enumerate}
\usepackage{graphicx}
\usepackage{float}
\usepackage{wrapfig}
\usepackage{color}
\usepackage[usenames,dvipsnames]{xcolor}
\usepackage[font=small,format=plain,labelfont=bf,up,textfont=it,up]{caption}
\usepackage{subfig}
\usepackage[colorlinks=false, pdfborder={1 0 0}]{hyperref}


\pagestyle{plain}
\pagenumbering{arabic}
\frenchspacing

\newcommand{\comments}[1]{}
\renewcommand{\baselinestretch}{1.55}

%Redefine the first level
\renewcommand{\theenumi}{\textbf{\alph{enumi})}}
\renewcommand{\labelenumi}{\theenumi}

\begin{document}

\title{\textbf{Verteilte Systeme SS2012 -- Übung 2}}
\author{Sebastian Menski (734272), Martin Ohmann (734801) \\ \texttt{\{menski,ohmann\}@uni-potsdam.de}}
\date{\today}

\maketitle

\section*{Aufgabe 2.1}

\begin{enumerate}

	\item Das Interface eth0 ist direkt an das private Netzwerk \texttt{10.3.5.0/24} angeschlossen und
	besitzt eine IP Adresse in diesem Netzwerk. Daher sollen alle Paket, welche ebenfalls an eine
	Adresse in diesem Netzwerk gesendet werden über das Interface eth0 verarbeitet werden. Der 2.
	Eintrag, beginnend mit \texttt{0.0.0.0} steht für die \textit{default} Route. Das bedeutet, dass
	alle Pakete, welche nicht an eine Adresse in dem durch die 1. Regel abgedeckten Netzwerk gesendet
	werden, an den Standard Gateway mit der Adresse \texttt{10.3.5.254} gesendet werden, welcher
	ebenfalls über das Interface eth0 erreichbar ist.

	\item Beide Programme nutzen ICMP Nachrichten. Ping sendet ein \textit{Echo Request}-Paket und
	erhält entsprechende Antwort ob ein Rechner verfügbar ist. Traceroute ermittelt alle Hops auf der
	Route zum Ziel. Dazu werden ebenfalls mehrere ICMP-Pakete gesendet, jedoch wird die Funktion der
	TTL genutzt. Das heißt, das erste gesendete Paket enthält eine TTL von 1, weshalb es schon vom
	ersten Hop auf der Route verworfen wird und dieser eine Fehlermeldung mit seiner Adresse
	zurückschickt. Dies wird mit steigender TTL wiederholt bis das Ziel erreicht wurde. Somit können
	alle Hops auf der Route erkannt werden.

	\item Mit dem Programm \texttt{route} können die IP-Routing-Tabellen des Kernels verändert werden,
	womit der komplette Netzwerk-Traffic des Rechners umgeleitet werden kann, was nicht jedem Nutzer
	möglich sein sollte. Das Programm \texttt{ifconfig} kann zum Konfigurieren der
	Netzwerkschnittstellen genutzt werden, welches ebenfalls eine Manipulation oder Störung des
	gesamten Netzwerk-Traffics ermöglicht, was ebenfalls nicht jedem Nutzer möglich sein sollte.

\end{enumerate}


\section*{Aufgabe 2.2}
\begin{table}[h]
	\begin{tabular}{|c|c|r|r|r|}\hline
		Netzklasse & Adressbereich               & Hostlänge & Netze      & Hosts pro Netz \\\hline
		Klasse A   &   0.0.0.0 – 127.255.255.255 & 24 Bit    &	      128	& 16.777.214     \\
		Klasse B   & 128.0.0.0 – 191.255.255.255 & 16 Bit    &	   16.384 &	    65.534     \\
		Klasse C   & 192.0.0.0 – 223.255.255.255 & 	8 Bit    &	2.097.152	&        254     \\\hline
	\end{tabular}
	\caption{Quelle: \url{http://de.wikipedia.org/wiki/Netzklasse}}
\end{table}


\section*{Aufgabe 2.3}
Da sich bei jedem Hop auf der Route eines IP Pakets die TTL verändert (dekrementiert wird) muss
jedesmal eine neue Checksum für das Paket ermittelt werden. Daher ist der Rechenaufwand kleiner,
wenn die Checksum nur über den Header berechnet wird. Außerdem führt dies dazu, dass sich die
Checksum pro Hop genau um 1 erhöht. Der Nachteil ist, dass die Payload von der überliegenden
Transportschicht überprüft werden muss. Dies kann jedoch erst am Ziel geschehen und wenn alle Pakete
erhalten wurden.


\section*{Aufgabe 2.4} 

\begin{itemize}

	\item Laut RFC müsste eine Überprüfung ob der \texttt{container} eine Teekanne (teapot) ist
	stattfinden, welche auch in Funktion \texttt{insert(container, water)} stattfinden könnte. Fehlt
	diese Überprüfung widerspricht dies jedoch nicht dem RFC, da diese Überprüfung bzw. deren
	Rückgabewert als \texttt{SHOULD} im RFC markiert ist, welches besagt, dass das Verhalten
	empfohlen ist, es jedoch gut begründete Ausnahmen geben kann. Ob es sich bei dem Code jedoch um
	eine gut durchdachte Implementation mit einer begründeten Ausnahme vom RFC handelt, ist auf Grund
	der mangelnden Kommentierung des Codes nicht nachzuvollziehen und eine Überprüfung auf ein
	Teekanne nach dem RFC ist zu empfehlen. Der Rückgabewert bzw. in diesem Fall der Response-Body
	ist nicht ersichtlich, jedoch auch beliebig zu wählen, da dieser im RFC als \texttt{MAY} markiert
	ist. Dies bedeute die Angaben sind optional und der Implementierung überlassen.

	\item Der Unterschied zwischen \texttt{SHALL} und \texttt{SHOULD} ist, dass Punkte, welche als
	\texttt{SHALL} markiert sind immer verpflichtend und erforderlich sind. Bei Punkten, welche als
	\texttt{SHOULD} markiert sind kann es begründete Ausnahmen geben, trotzdem sind sie im Normalfall
	gefordert und sollten nur in gut durchdachten Situationen verändert werden.

\end{itemize}

\end{document}
